\ProvidesFile{seminar06.tex}

\section{Семинар 6}

\subsection{Задача 1}

Докажи, что количество перестановок $n$ элементов, при которых ни один элемент не
остается на своем месте (то есть таких перестановок $\sigma$, что $\forall k \leadsto \sigma(k) \neq  k$, равно:
\[
n! \cdot (1 - 1 + \frac{1}{2!} - \frac{1}{3!} + \ldots + (-1)^n \cdot \frac{1}{n!})
\]

\begin{proof}
Давайте посчитаем количество перестановок, где есть неподвижный элемент.

Пусть $N_{i_1, \ldots i_n}$ - это число перестановок, где $i_1, \ldots, i_n$ -- неподвижные элементы\footnote{То есть те перестановки, где $\sigma(i) = i \ \forall i $}.

Ясно, что $N_i = (n-1)!$, $N_{ij} = (n-2)!, \ldots$

Мы хотим посчитать $|A_1 \cup A_2 \ldots A_n|$.

По формуле включений-исключений это равно:
\[
|A_1 \cup A_2 \cup \ldots \cup A_n| = \sum_{i=1}^nN_i - \sum_{i<j}{N_{ij}} + \sum_{i<j<k}{N_{ijk}} - \ldots + (-1)^n N_{1\ldots n} =
\]
\[
= n \cdot (n-1)! - C^2_n \cdot (n-2)! + \ldots + (-1)^n \cdot 1 = \frac{n!}{1} - \frac{n!}{2} + \frac{n!}{3} + \ldots + (-1)^n \frac{n!}{n!} = n! \cdot (1 - \frac{1}{2}  + \ldots + (-1)^n \frac{1}{n!})
\]

Всего перестановок на $n$ элементах $n!$, поэтому чтобы получить итоговый ответ, нужно лишь вычесть эту сумму из $n!$.

\textbf{Ответ:}
\[
n! \cdot (1 - 1 + \frac{1}{2!} - \frac{1}{3!} + \ldots + (-1)^n \cdot \frac{1}{n!})
\]
\subsection{Задача 2}

Ясно, что нужно просто поделить на $n!$ ответ из прошлой задачи, так как нам не важен порядок людей, которые взяли свою бумажку.

\textbf{Ответ:} $1 - 1 + \frac{1}{2!} - \frac{1}{3!} + \ldots + (-1)^n \cdot \frac{1}{n!}$

\subsection{Задача 3} 

\begin{enumerate}
\item Докажи, что: $$\stirling{n}{1} = (n-1)!$$
\item Найди $\stirling{4}{2}$
\end{enumerate}

\subsubsection{Пункт А}

\begin{proof}
Заметим, что обе штуки можно интерпретировать как число хороводов (то есть $n$ людей, которые идут в одном направлении). В левой части мы просто выбрали 1 цикл и это однозначно задает, кто за кем идет. В правой части мы взяли любого человека и для него решили, за кем из оставшихся $n - 1$ людей он идёт. Вот и получается, что части равны.
\end{proof}

\subsubsection{Пункт Б}

Нас устраивают перестановки вида $(*)(***)$ и $(**)(**)$.

Перестановок первого вида $4 \cdot 2!$ (выберем кто идет в одиночный цикл, а для оставшихся двух воспользуемся результатом из пункта А)

Перестановок второго вида $\frac{C^2_{4}}{2}$ (выбрать двух из четырех в первый цикл, и нам не важна перестановка циклов).

\textbf{Ответ:} $4 \cdot 2! + \frac{C^2_{4}}{2}$

\end{proof}

\subsection{Задача 4}

Докажи, что:
\[
\stirling{n+1}{k} = n \cdot \stirling{n}{k} + \stirling{n}{k-1}
\]

\begin{proof}
Пусть есть один выделенный элемент.

Ясно, что второе слагаемое соответствует случаю, когда выделенный элемент находится в одиночном цикле (на оставшихся $n$ элементах нам остается лишь распределить их по $k - 1$ циклу).

Подумаем над первым слагаемым. Там написано разбиение $n$ элементов на $k$ циклов. Теперь попробуем вставить выделенный элемент в какой-нибудь из циклов. 

В любой цикл длины $m_i$ можно поставить этот выделенный элемент на любое из $m _i+ 1$ мест, но на самом деле случай, когда выделенный элемент стоит в начале и конце цикла -- это одно и то же. Поэтому, на самом деле в любом цикле длины $m_i$ есть $m_i$ способов. 

Ясно, что выделенный элемент можно поставить в любой из циклов, поэтому для этого слагаемого есть $\sum_{i} m_i$ способов. Но суммарная длина циклов равна $n$. Вот и получается $n$ способов.
\end{proof}

\subsection{Задача 5}

Пусть $s(n,k)$ обозначает коэффициент при $x^k$ в многочлене 
\[
x(x-1)(x-2)\ldots(x-n+1)
\]
Докажи, что $\stirling{n}{k} = |s(n,k)|$

\begin{proof}
Помянем все минусы на плюсы, так как у нас есть модуль, то это ничего не поменяет.

Введём обозначения:

$x^{\overline{1}} = x$

$x^{\overline{2}} = x(x+1)$

$\ldots$

$x^{\overline{n}} = x(x+1)(x+2)\ldots(x+n-1)$\footnote{Такие обозначения называются восходящим факториалом (rising factorial) или Символом Похгаммера (Pochhammer symbol)}

Будем доказывать это утверждение по индукции. 

База при $n = 1$ очевидна.

Покажем индукционный переход.

Заметим, что $(x+(n-1))x^k = x^{k+1} + (n-1)x^k$. Но: 
\[
x^{\overline{n}} = (x+(n-1))x^{\overline{n-1}}
\]

В силу индукционного предположения:
\[
x^{\overline{n}} = \lr{x+\lr{n-1}}\sum_{k=1}^{n-1}{\stirling{n-1}{k}}x^k = \sum_{k=1}^{n-1}{\stirling{n-1}{k}x^{k+1}} + \sum_{k=1}^{n-1}{\lr{n-1}\stirling{n-1}{k}} \cdot x^k = \sum_{k=1}^{n}{\stirling{n-1}{k-1}x^{k}} + \sum_{k=1}^{n-1}{\lr{n-1}\stirling{n-1}{k}} =
\]
\[
= \sum_{k=1}^{n}\lr{\stirling{n-1}{k-1} + \lr{n-1}\stirling{n-1}{k}}x^k
\]

Воспользуемся утверждением из прошлой задачи и получим, что:

\[
\sum_{k=1}^{n}\lr{\stirling{n-1}{k-1} + (n-1)\stirling{n-1}{k}}x^k = \sum_{k=1}^{n}\stirling{n}{k}x^k
\]
Это доказывает индукционный переход.
\end{proof}

\subsection{Задача 6}

Докажи, что $S(n,k) = S(n - 1,k - 1) + kS(n - 1,k).$

\begin{proof}
Первое слагаемое соответствует тому, что выделенный элемент занял одну новую коробку. На оставшихся $n  1$ элементах остается $k - 1$ коробка

Второе слагаемое соответствует тому, что мы распределили $n - 1$ элемент по $k$ коробкам, а потом выделенный засунули в одну из них.

Вот и получается, что это тождество верно.
\end{proof}

\subsection{Задача 7}
Докажи, что $S(n,k) = k^n - C^1_k \cdot (k-1)^n + C^2_k \cdot (k-2)^n - \ldots + (-1)^{k-1} \cdot C^{k-1}_{k} \cdot 1^n$

\begin{proof}
Это неслучайно похоже на формулу включений-исключений.

Давайте посчитаем величину $|A_1 \cup A_2 \ldots  \cup A_k|$, где $A_i$ -- это множество способов, при которых $A_i$ пуста.

Ясно, что $|A_i| = \frac{(k-1)^n}{k!}$, так как это интерпретируется так: у нас остается $k-1$ коробка и мы для каждого из $n$ элементов можем положить любой из $n$ элементов лишь в одну из $k-1$ коробок. Но мы не различаем коробки. Вот и возникает деление.

Аналогично: $|A_i \cap A_j| = \frac{(k-2)^n}{k!}$

Ну и так далее вплоть до $|A_1 \cap A_2 \ldots \cap A_k| = 0$ (если ни в какую коробку нельзя ничего положить, то у нас 0 способов разложить $n$ элементов по этим коробкам). 

Тогда по формуле включений-исключений:
\[
|A_1 \cup A_2 \ldots  \cup A_k| = \frac{k^n}{k!} - C^1_k \cdot \frac{(k-1)^n}{k!} + C^2_k \cdot \frac{(k-2)^n}{k!} - \ldots + (-1)^{k-1} \cdot C^{k-1}_{k} \cdot 1^n = \frac{1}{k!} \cdot \sum_{i=0}^{k-1}{(-1)^i\cdot C^{k-i}_{k} \cdot (k-i)^n}
\]
\end{proof}

\subsection{Задача 8}

Пусть $X, Y$ – конечные множества, причем в первом из них $k$ элементов, во втором $n$
элементов. Напомним, что сюръекцией из множества $X$ в множество $Y$ называется такая функция $f$ из $X$ в $Y$, что для любого $y \in Y$ найдется $x \in X$, такой что $f(x) = Y.$ Сколько существует различных сюръекций из $X$ в $Y$?

Пусть элементы в $Y$ -- это $n$ коробок, а элементы в $X$ -- это $k$ элементов, которые мы хотим распределить по $n$ коробкам. Тогда ясно, что нас интересует число $S(k, n)$. Но число Стирглинга второго рода не различает коробки между собой, а для нас очень важно, в какой именно элемент из $Y$ отображаться. Поэтому нужно домножить на $n!$.

\textbf{Ответ:} $S(k, n) \cdot n!$

\subsection{Задача 9}

Докажи, что $\displaystyle B_n = \sum_{k=0}^n{S(n,k)}$

\begin{proof}
Слагаемое с $k = 0$ можно сразу выкинуть, так как очевидно, что оно равно нулю (0 способов разложить $n$ элементов по $k$ коробкам).

Остальные слагаемые с $k \neq 0$ интерпретируются так: мы задаем $k$ классов эквивалентности и для этого числа классов эквивалентности смотрим все способы разбить $n$ элементов на эти классы эквивалентности, чтобы ни один из них не был пуст.
\end{proof}