\ProvidesFile{seminar07.tex}

\section{Семинар 7}

\subsection{Задача 1}
Существует ли 1000000 подряд идущих натуральных чисел, среди которых ровно 7
простых?

\begin{proof}
Ясно, что среди первого миллиона чисел есть больше 7 простых чисел.

Но при этом среди чисел $100001! + 2, 100001! + 3, \ldots, 100001! + 1000001$ нет ни одного простого числа. Действительно: первое число делится на 2, второе -- на 3, $\ldots$, последнее -- на 1000001.

Тогда по принципу дискретной непрерывности двигая <<окошко>> из миллиона простых чисел обязательно найдется миллион простых чисел, среди которых ровно 7 простых чисел.
\end{proof}

\subsection{Задача 2}
На клетчатой доске $100 \times 100$ половина клеток белые, а половина -- черные. Докажи,
что можно разрезать ее по границам клеток на две части с равным числом черных
клеток.

\begin{proof}
Будем добавлять по одной клетке, начиная с угловой. Количество черных клеток увеличивается либо на 0, либо на 1. В начале число черных клеток либо 0, либо 1. В конце их $5000$ По принципу дискретной непрерывности будет момент, в который будет 2500 черных клеток. 
\end{proof}

\subsection{Задача 3}
\textbf{[Всерос, закл 2016, 10-11 класс]} По кругу стоят $n$ мальчиков и $n$ девочек. Назовём пару из мальчика и девочки хорошей, если на одной из дуг между ними стоит поровну мальчиков и девочек (в частности, стоящие рядом мальчик и девочка образуют хорошую пару). Оказалось, что есть
девочка, которая участвует ровно в 10 хороших парах. Докажи, что есть и мальчик,
который участвует ровно в 10 хороших парах.

Решение взято с сайта \href{https://problems.ru/view_problem_details_new.php?id=65713}{problems.ru}

Заметим, что на любой дуге между членами хорошей пары поровну девочек и мальчиков.
Пусть $D$ –- девочка, участвующая в 10 хороших парах. Обозначим всех детей по часовой стрелке $K_1, K_2, \ldots, K_{2n}$ так, что $K_1$ – это $D$, и продолжим нумерацию циклически (например,  $K_0 = K_{2n}$  и  $K_{2n+1} = K_1$).  При $i = 1, 2, ..., 2n$  обозначим через $d_i$ разность между количествами девочек и мальчиков среди $K_1, K_2, ..., K_i$; в частности, $d_1 = 1 - 0 = 1$ и  $d_{2n} = 0$ (поэтому можно продолжить эту последовательность, полагая  $d_{2n+1} = d_1$ и т. д.). Девочка $D$ образует с $K_i$ хорошую пару тогда и только тогда, когда  $d_i = 0$  и  $K_i$ – мальчик, то есть  $d_i = 0$  и  $d_{i-1} = 1$.  Итак, найдутся ровно 10 индексов $i$ с такими свойствами.

Рассмотрим любого мальчика  $M = K_s$,  образующего с $D$ хорошую пару; тогда  $d_s$ = 0  и  $d_{s-1} = 1$.  Аналогично получаем, что мальчик $M$ образует хорошую пару с $K_i$ ровно тогда, когда $d_s = d_{i-1}$ и $K_i$ – девочка (то есть  $d_{i-1} = 0$ и $d_i = 1$).

Заметим, что любые два числа $d_i$ и $d_{i+1}$ отличаются на единицу. Разобьём их на группы последовательных чисел, не меньших единицы, и группы последовательных чисел, не больших нуля. Тогда при обходе круга по часовой стрелке "переходов" из первых групп во вторые будет столько же, сколько и "переходов" из вторых групп в первые. Значит, у $M$ столько же хороших напарниц, сколько у $D$ хороших напарников. Это и требовалось доказать.

\subsection{Задача 5}

Клетки шахматной доски 8 × 8 занумерованы числами от 1 до 32 так, что каждое число использовано дважды. Доказать, что можно выбрать 32 клетки, занумерованные
разными числами, так, что на каждой вертикали и каждой горизонтали найдется хотя
бы по одной выбранной клетке.

\begin{proof}
Ясно, что всего выборов клеток $2^{32}$. А сколько <<плохих>> выборов? 

Пусть в 1-ом столбце ничего не выбрано. Значит, мы не выбрали ни одно число из этого столбца. Тогда у меня лишь свобода на оставшихся 24 числах. То есть таких способов $2^{24}$

Аналогично для оставшихся 7 столбцов и 8 строк. Тогда получается, что этих <<плохих>> способов $16 \cdot 2^{24} = 2^{28}$

Плохих способов меньше чем всего способов. Значит есть и хороший способ.
\end{proof}

\subsection{Задача 7}

На турнир приехали $n$ игроков. Каждая пара игроков, согласно регламенту турнира,
должна провести одну встречу (ничьих не бывает). Доказать, что при некотором $n$
игроки могли сыграть так, что для каждого множества из $k$ игроков найдется игрок,
победивший их всех

Ясно, что игр было сыграно $C^2_n$.

Но тогда каждой игры есть 2 исхода: победил первый или победил второй.

Значит, существует всего $\displaystyle 2^{C^2_n}$ турниров.

Пусть $A_k$ -- множество всех турниров, когда для  набора игроков $K$ ($|K| = k$) не существует игрока из $K$, который победил всех других игроков $K$.

Заметим, что $|\cup A_k| = C^k_n \cdot (2^k - 1)^{n - k}$.

Теперь единственное, что нужно понять: почему это число меньше, чем общее число турниров.

Заметим, что $C^k_n \cdot (2^k - 1)^{n - k} < 2^{n-1} (2^k)^{n-k} = 2^{-k^2 + nk + (n - 1)} = 2^{-\frac{n^2}{4} + n - 1}$

Ну и по индукции там ещё с помощью технической возни доказывается, что эта штука и вправду меньше, чем общее число турниров.