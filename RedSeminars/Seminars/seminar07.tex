\ProvidesFile{seminar07.tex}

\section{Семинар 7}

\subsection{Задача 1}

Есть 100 камней, выложенных в порядке возрастания весов. За какое наименьшее
число взвешиваний на чашечных весах без гирь можно проверить или опровергнуть
утверждение: <<Любые пять камней вместе тяжелее любых трех>>?

\begin{solution*}
За одно. Достаточно взять 5 самых легких камней и 3 самых тяжелых.
\end{solution*}

\subsection{Задача 2}

Есть несколько мешков, в каждом из которых достаточно много монет. В одном мешке монеты фальшивые, а во всех других настоящие. Известен вес настоящей монеты
и известно, что фальшивая монета на 1 грамм легче настоящей. Есть весы, которые
показывают точный вес любого набора монет. Как за одно взвешивание определить
мешок с фальшивыми монетами?

\begin{solution*}
Берем из 1-го мешка одну монету, из 2-го мешка 2 монеты, $\ldots$, $n$-го мешка $n$ монет,

Путь $X$ -- вес настоящей монеты. Тогда если бы все монеты были бы настоящими, то вес всех монет был бы $\displaystyle \frac{n(n+1)}{2} \cdot X$. Но на самом деле вес будет отличаться на $l$, так как в $l$-ом мешке лежат фальшивые монеты. Но тогда с помощью подсчёта разницы между ожидаемым и фактическим весом мы можем вычислить $l$.
\end{solution*}

\subsection{Задача 3}

Рассмотрим лексикографический порядок на всех сочетаниях из 10 по 5 элементов.

а) Какой номер будет иметь сочетание $\left\{2,3,5,9,10\right\}$?

б) Какое сочетание будет под номером 127?

\begin{solution*}
Внутри сочетания числа не упорядоченны между собой, поэтому введение лексикографического порядка лишь упорядочивает их внутри сочетания.

\subsubsection{Пункт Б}

Сначала легче его решить его.

Пусть мы зафиксировали первое число единицей. Тогда для оставшихся 4 <<мест>> внутри сочетания остаётся 9 чисел. Тогда таких сочетаний (с единицей на первом месте) ровно $C^4_9 = \frac{9!}{4!5!} = 126$. То есть 126 сочетаний начинаются с единицы. Ясно, что 127 сочетание -- это самое маленькое в смысле лексикографического порядка сочетание, которое начинается двойки. Поэтому ясно, что это будет сочетание $\lr{2, 3, 4, 5, 6}$

\textbf{Ответ:} $\lr{2, 3, 4, 5, 6}$ 

\subsubsection{Пункт А}

Ясно, что 2 и 3 -- это минимальные первые числа для сочетаний, начинающихся с двойки.

Посчитаем число сочетаний, у которых потом идет 4. Ясно, что достаточно выбрать на 2 оставшихся места любые из 6 чисел. То есть существует $C^2_6 = 15$ способов.

Посчитаем число сочетаний, у которых потом идет 5. Ясно, что достаточно выбрать на 2 оставшихся места любые из 5 чисел. То есть существует $C^2_5 = 10$ способов.

Заметим, что 9 и 10 -- это максимально возможные последние два числа в сочетании.

Поэтому ответ на задачу -- 126 + 15 + 10 = 151.
\end{solution*}

\textbf{Ответ:} 151 
 
\subsection{Задача 4}

Илья хочет сообщить Владе некоторое слово из 10 букв, в котором четыре раза встречается буква $A$, три раза встречается буква $B$ и три раза встречается буква $C$. Однако канал связи между ними позволяет передавать лишь нули и единицы. Ребята
хотят заранее договориться о способе кодирования слов так, чтобы в худшем случае
использовать минимально возможное число бит для передачи информации.

а) Разработайте такой способ кодирования. Какое количество бит потребуется для
кодирования каждого слова при оптимальном способе кодирования?

б) Какой двоичный код будет соответствовать слову $BAABCCACAB$?

\begin{solution*}
Посчитаем, сколько всего существует слов. Ясно, что нужно выбрать 4 места из 10 для A и из 6 оставшихся 3 места для B. На оставшиеся места встают C. Тогда всего слов $C^4_{10} \cdot C^3_6 = 4200$.

Ясно, что слов длины $k$ для алфавита только из двух символов ровно $2^{k}$

Заметим, что $2^{12} = 4096$. То есть 12 бит не хватит, а вот 13 бит уже хватит, так как $2^{13} = 8192$.

\subsubsection{Пункт А}

Достаточно ввести лексикографический порядок на этих сочетаниях и отобразить номер каждого сочетания в двоичную систему.

\subsubsection{Пункт Б}

Нужно применить рассуждения, аналогичные третьей задаче. Не хочется расписывать это. Ответ на этот пункт, кстати, будет зависеть от того, как именно мы выберем наш лексикографический порядок. Так как буквы $A, B, C$ можно по разному сравнить между собой и это даст разные порядки (и следовательно, разные ответы)

\end{solution*}

\subsection{Задача 5}

Фокусник решил показать зрителям фокус с угадыванием карты. В колоде у зрителя
есть несколько карт, он загадывает одну из них. Фокусник раскладывает все карты на
4 стопки и узнает у зрителя, в какой стопке оказалась задуманная карта. При каком
наибольшем количестве карт можно наверняка определить задуманную карту за 3
вопроса?

\textbf{Ответ: 64}

\textbf{Пример:} Ясно, что с 64 картами всё получится, нужно лишь делить каждый каждый раз стопку на 4. За 3 хода ровно и справимся.

\textbf{Оценка:} Закодируем ответы зрителя в четверичной системе. Например если он выбирал стопки в порядке 1, 3, 4, 2, то этому будет соответствовать число $0231_4$. Таких последовательностей 64, поэтому если карт будет 65 и больше, то по принципу Дирихле какой-то последовательности будут соответствовать 2 карты.

\subsection{Задача 6}

Имеется 12 на вид неотличимых монет, из которых одна фальшивая. Все настоящие
монеты весят одинаково, а фальшивая отличается от настоящих по весу, но неизвестно, в какую сторону (легче или тяжелее). Имеются чашечные весы, на которых за одно
взвешивание можно сравнить любые две группы монет по весу. За какое минимальное число взвешиваний гарантированно можно найти фальшивую монету и узнать,
легче она или тяжелее, чем настоящие?

\textbf{Ответ:} 3

\textbf{Оценка:} Упростим нашу задачу, и допустим мы сразу знаем, тяжелее ли фальшивая монетка или легче. Даже в таком случае нам нужно минимум $\lceil\log_3{12}\rceil = 3$ взвешивания. Поэтому за два и меньше точно не получится.

\textbf{Пример:}

Сначала я опишу алгоритм, а потом объясню, почему он работает.

\textbf{Алгоритм:}

Нумеруем монеты в троичной системе вот так:
\[
001 \quad 010 \quad 011 \quad 012
\]
\[
112 \quad 120 \quad 121 \quad 122
\]
\[
200 \quad 201 \quad 202 \quad 220
\] 

Теперь сначала я опишу, какие взвешивания я буду производить, а затем напишу, какую информацию от каждого взвешивания я буду запоминать.

Сначала я беру монетки с 1-ой цифрой 0 на левую чашу весов, с 1-ой цифрой 2 на правую чашу весов, с 1-ой цифрой 1 не кладу на весы.

Затем я беру монетки с 2-ой цифрой 0 на левую чашу весов, с 2-ой цифрой 2 на правую чашу весов, с 2-ой цифрой 1 не кладу на весы.

Затем я беру монетки с 3-ей цифрой 0 на левую чашу весов, с 3-ей цифрой 2 на правую чашу весов, с 3-ей цифрой 1 не кладу на весы.

После очередного взвешивания я приписываю себе на листочек по одной цифре по следующему правилу:

Если перевесила левая чаша весов, то я записываю себе на листочек цифру 0.

Если перевесила правая чаша весов, то я записываю себе на листочек цифру 2.

Если чаши весов равны, то я записываю на листочек цифру 1.

В результате я получу код из трёх цифр.

Если полученный код присутствует в списке занумерованных монет (например, 112), то монета с этим номером фальшивая и тяжелая.

Если полученного кода нет в списке, то фальшивая монета обязательно лёгкая. В полученном коде нужно заменить все нули на двойки, а все двойки на нули. Тогда мы обязательно получим номер фальшивой монеты.

\textbf{Теперь объяснение, что это вообще был за алгоритм и почему он работает:}

Смотрите, нумерация монет выбрана оооочень не случайно. Как минимум стоить заметить, что на каждой из трёх позиций каждая из цифр 0, 1 или 2 встречается ровно по 4 раза.

На каждом ходе мы имеем 3 кучки по 4 монеты. Причём на кучи мы разбиваем по цифрам на соответствующих разрядах. Значит на каждом шаге мы записываем цифру с тяжелых куч. 

Если фальшивая монета была тяжелой, то мы выписали её код, так как мы выписывали цифры самых тяжелых куч. В частности, это объясняет, что для тяжелой монеты мы не можем получить код не из списка (и следовательно никаких замен нулей с двойками в этом случае нет и быть не может). Это лёгкий случай.

Если фальшивая монета была легкой, то этот случай сильно сложнее. Нужно объяснить: 1) почему мы получим код не из списка 2) почему замена нулей с двойками даст нужный код фальшивой монеты из списка.

Важное свойство нашей нумерации монет состоит в том, что при замене в любом номере нулей на двойки, а двоек на нули, мы обязательно получаем код, которого нет в списке.

Что происходит при взвешиваниях: если легкая монета не попадает на весы, то мы записываем единицу. Если легкая монета попадает на весы, то мы записываем ноль вместо двойки и двойку вместо нуля. То есть как раз получаем код не из списка. И при повторной такой замене мы получим нужный номер нашей легкой монеты.
\subsection{Задача 7}

Из 11 шаров два радиоактивны. Про любую кучку шаров за одну проверку можно
узнать, имеется ли в ней хотя бы один радиоактивный шар (но нельзя узнать, сколько
их). За какое наименьшее число проверок можно гарантированно выявить оба радиоактивных шара?

\textbf{Ответ:} 7

\textbf{Пример:} Разбиваем все шары на 5 групп по 2 и остаётся ещё 1 шар.

Проверяем для первых пяти групп, есть ли в каждой из групп радиоактивный элемент. 

Если получилось 2 группы с радиоактивными элементами, то в каждой из групп достаточно проверить по одному элементу (ещё 2 проверки).

Если получилась 1 группа с радиоактивным элементом, в этой группе достаточно проверить один элемент, и взять ещё тот, который мы не проверяли (ещё 1 проверка).

\textbf{Оценка:} Оценки устроены в построении контр-примеров для каждого числа шаров в первом вопросе.

Покажу на примере не более двух шаров в первом вопросе.

Допустим, мы потратили вопрос, чтобы узнать, что в группе из двух шаров нет радиоактивного. Тогда у нас осталось 9 шаров и не более пяти вопросов. 

Вариантов того, кто из них может быть радиоактивен ровно $C^2_9 = 36$. Комбинаций ответов <<да>> или <<нет>> на 5 вопросов ровно $2^5 = 32$.

$C^2_9 > 2^5$. Следовательно, вариантов для радиоактивности шаров больше, чем потенциальных комбинаций ответов.

Аналогично делается для любого другого числа шаров в первом вопросе. Это не очень идейно.

\subsection{Задача 8}

Есть 100-этажное здание. Известно, что если яйцо сбросить с высоты $N$-го этажа
(или с большей высоты), то оно разобьется. Если его бросить с любого меньшего этажа, оно не разобьется. У нас есть всего два яйца, и если они оба разобьются, бросать
больше будет нечего. За какое минимальное количество бросков наверняка получится определить $N$?

\textbf{Ответ:} 14

Дикий баян. Есть \href{https://youtu.be/f2VLdwU9xc4?si=poEv93D02R2l0ry1&t=2396}{видео-разбор у Трушина}. Вставлю текстовый разбор с первого сайта в гугле.

Обратите внимание, что независимо от того, с какого этажа мы бросаем яйцо №1, бросая яйцо №2, необходимого использовать линейный поиск (от самого низкого до самого высокого этажа) между этажом «повреждения» и следующим наивысшим этажом, при броске с которого яйцо останется целым. Например, если яйцо №1 остается целым при падении с 5-го по 10-й этаж, но разбивается при броске с 15-го этажа, то яйцо №2 придется (в худшем случае) сбрасывать с 11-го,12-го,13-го и 14-го этажей.

Предположим, что мы бросаем яйцо с 10-го этажа, потом с 20-го…

Если яйцо №1 разбилось на первом броске (этаж 10-й), то нам в худшем случае приходится проделать не более 10 бросков.
Если яйцо №1 разбивается на последнем броске (100-й этаж), тогда у нас впереди в худшем случае 19 бросков (этажи 10-й, 20-й, …, 90-й, 100-й, затем с 91-го до 99-го).
Это хорошо, но давайте уделим внимание самому плохому случаю. Выполним балансировку нагрузки, чтобы выделить два наиболее вероятных случая.

В хорошо сбалансированной системе значение Drops(Egg1) + Drops(Egg2) будет постоянным, независимо  от того, на каком этаже разбилось яйцо №1.
Допустим, что за каждый бросок яйцо №1 «делает» один шаг (этаж), а яйцо №2 перемещается на один шаг меньше.
Нужно каждый раз сокращать на единицу количество бросков, потенциально необходимых яйцу №2. Если яйцо №1 бросается сначала с 20-го, а потом с 30-го этажа, то яйцу №2 понадобится не более 9 бросков. Когда мы бросаем яйцо №1 в очередной раз, то должны снизить количество бросков яйца №2 до 8. Для этого достаточно бросить яйцо №1 с 39 этажа.
Мы знаем, что яйцо №1 должно стартовать с этажа X, затем спуститься на X-1 этажей, затем — на X-2 этажей, пока не будет достигнуто число 100.
Можно вывести формулу, описыващее наше решение:  X + (X – 1) + (X – 2) + … + 1 = 100 -> X = 14.
Таким образом, мы сначала попадаем на 14-й этаж, затем на 27-й, затем 39-й. Так что 14 шагов — худший случай.

Как и в других задачах максимизации/минимазиции, ключом к решению является «балансировка худшего случая».

Разбор взят из книги Гейл Л. Макдауэлл «Cracking the Coding Interview» (есть в переводе).

\subsection{Задача 9}

Имеется $n$ различных по весу монет. За одно взвешивание можно сравнить любые
две монеты и узнать, какая из них тяжелее. Требуется найти вторую по весу монету
среди данных $n$ монет.

а) Докажите, что существует алгоритм, который гарантированно решает эту задачу
за $n$ + $log_2{n} + O(1)$ сравнений.

б) Докажите, что более оптимального алгоритма не существует.

\subsubsection{Пункт А}

Алгоритм такой: выставляем все эти $n$ монет в ряд, из пары 1-2 выбираем победителя, из пары 3-4 выбираем победителя, $\ldots$, из последней пары выбираем победителя. Если осталось число, то переносим его на следующий этап.

Делаем так до тех пор, пока не останется одно число. На это потребуется $n - 1$ операция.

А затем проходимся по дереву тех, кто проигрывал у первой монетки. Там не более чем $\log_2{n}$ монеток. Нужно из них выбрать самую тяжелую. Ясно, что на это потребуется $\log_2n - 1$ операция.

\subsubsection{Пункт Б}
\begin{proof}
Если я знаю, что какая-то монетка является второй по весу, то уж я как минимум уверен, что она меньше по весу, чем первая монетка. Но тогда я точно знаю, какаю монетка является первой по весу (иначе как я был уверен в том, что мы меньше, чем первая по весу монетка?)

Значит, мне нужно как минимум $n - 1$ сравнение потратить на то, чтобы выявить самую тяжелую монетку (доказательство есть в лонгриде/было на лекции).

Но тогда после $n - 1$ сравнения мне нужно сравнить элементы, с которыми сравнивалась самая крупная монетка. Их уж точно не меньше, чем $\log_2n - 1$, иначе у нас не было бы достаточно информации для вывода о том, что самая тяжелая монетка и в правду самая тяжелая.

Вот и получается оценка $n + \log_2n + O(1)$
\end{proof}